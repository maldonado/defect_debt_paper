\documentclass[conference]{IEEEtran}

\usepackage{amssymb,amsmath}
\usepackage{wrapfig}
\usepackage{multirow}
\usepackage{graphicx}
\usepackage{algorithm}
\usepackage{algorithmic}
\usepackage{times}
\usepackage{cite}
\usepackage{url}
\usepackage{booktabs}
\usepackage{subfigure}
\usepackage{fancybox}
\usepackage{color}
\usepackage{array}
\usepackage{subfigure}
\usepackage{balance}
\usepackage{epstopdf}
\usepackage{array}

\newcommand{\everton}[1]{\textcolor{blue}{{\it [Everton: #1]}}}
\newcommand{\todo}[1]{\colorbox{yellow}{\textbf{[#1]}}}


\newcommand{\conclusionbox}[1]{%
	\vspace{2mm}
	\framebox[0.45\textwidth][c]{%
		\parbox[b]{0.42\textwidth}{%
			{\it #1}
		}
	}
	\vspace{2mm}
}

\newcommand{\rqi}{\textbf{RQ1- How to characterize Defect Debt in a software project?}}
\newcommand{\rqii}{\textbf{RQ2- Defect Debt has a negative impact in feature addition?}}
\newcommand{\rqiii}{\textbf{RQ3- It is possible to create a model to Defect Debt?}}

\begin{document}
\title{On Characterizing and Modeling Defect Technical Debt}

\author{\IEEEauthorblockN{Everton da S. Maldonado}

\IEEEauthorblockA{Department of Computer Science and Software Engineering\\Concordia University,
Montreal, Canada\\
\url{everton.maldonado@gmail.com}}}

\maketitle

\begin{abstract}
During the development and maintenance of a software system, developers face unpredictable difficulties or pressures, and in many cases are forced to apply unconventional solutions to overcome these difficulties. For example, they might adopt insufficiently tested or temporary solutions (i.e., workarounds and hacks), neglect good design practices, and introduce inaccurate or incomplete documentation
due to time constraints and pressure to meet deadlines. This phenomenon has been explained through the metaphor of Technical Debt.

Prior work has shown that one of the most impacting types of technical debt is design debt and that code comments embedded in the code can be used to detect \emph{self-admitted} technical debt. Therefore, in this paper our main goal is to study 
More specifically, we derive comment patterns that can be used to detect  Then, we perform a case study to determine the effectiveness of our approach at detecting  We also compare the effectiveness of our approach to prior approaches that use code smells to detect design technical debt and quantify how much of the self-admitted design debt can be automatically refactored with refactoring tools. We suggest 176 different comment patterns that can be used to detect  Our approach can achieve precision and recall values between 74.07-96.30\% and 10.87-83.87\%, respectively. We also show that our approach detects design technical debt that is different from alternative state-of-the-art techniques used for finding design technical debt. Lastly, our findings also show that 24.58\% of the is detected in the form of refactoring opportunities by a state-of-the-art refactoring recommendation tool. 
\end{abstract}

\IEEEpeerreviewmaketitle

\section{Introduction}
\label{sec:introduction}
Software system becomes more important each passing day and the effort to produce good and successful software projects increases at the same pace. In an ideal world a project  should be delivered on time, inside the budget and with high quality. We know by experience that is very  difficult to balance these elements, and they are mostly likely to be indirectly proportional. Quality is very important though, and many approaches have been proposed to ensure software quality. However there is always a tradeoff between quality and and resources. For example, developers often take shortcuts to meet release deadlines, fix last minute bugs, etc.

A new area that studies this phenomena is Technical Debt. The term technical debt is used to express faults and non-optimal solutions that are taken conscientiously in a software project in order to achieve short term goals. Usually  this decisions allows the project to move faster towards its goal but there will be an increased cost to maintain this software in the long run. Prior work \cite{Lim2012Software} showed that technical debt is unavoidable and that there is different types of technical debt, like design debt, test debt, documentation debt, defect debt, etc.  

The technical debt metaphor is widely used in industry and the research in this area is expanding as well. However, in topics like defect debt, different works has a divergent opinion. Some studies says that defect technical debt does not exists \cite{Kruchten2013GSOFT} as in the other hand, there are studies supporting the existence of defect technical debt \cite{Seaman201125}. We argue that there are different kinds of defects in a project, and that some of them can be characterized as defect technical debt, although not all of defects are necessarily  a defect technical debt. 

Our goal is to understand, characterize and model Defect Debt, being able to distinguish Defect Debt from other Bugs.

In this paper, we extract more than 139,000 commits and analyze more than 67,000 bug reports from one large open source software. We performe a empiracal study on Google Chrome to answer our research questions. 

Fisrt, we analyze 25 releases from Chrome grouping the number of reported bugs and fixed bugs for each one of them. Based on the result of this analysis we found a pattern that can categorize Defect Debt and Bugs. We found that Bugs are more urgent defects, that are fixed in the same release as reported or in the immediatly next one. However, Defect debt lingers in the system for a long time, but eventually get fixed. In the analyzed project the avarage percentage of Bugs are of 69.55\% and Defect Debt is of 30.44\%.

Second, we analyze the total number of open bugs in the system and the number of Defect Debt. Then we classify the commits accordingly with their intention. We use this information in to undertand how Defect Debt impacts the addition of new features. Based on our result we find that there is no direct negative relation between the number of Defect Debt and the addition of new features. 

Third, we collect several metrics and propose a model to Defect Debt, we find that it is possible to build such a model, but it is necessary to treat the data in a more granular level (i.e., file level). 

The rest of the paper is organized as follows: Section~\ref{sec:related_work} presents the related work, followed by Section~\ref{sec:approach} that details our approach. We present our case study results in Section~\ref{sec:results}. The threats to validity of our work are discussed in Section~\ref{sec:threats_to_validity}. Section~\ref{sec:conclusion} lists the ours conclusions and future work. 

\section{Related Work}
\label{sec:related_work}
Chen et al. \cite{Chen2014MSR} conduct their work in dormant bugs. They classify dormant bugs as 

Our work differs from previous work because a Defect Debt is a know issue to the development team whereas a Dormant Bug is a defect that is hidden for a long period of time and eventually it crash. In other words Defect Debt is know option not an accident.

\section{Approach}
\label{sec:approach}
\begin{figure*}[thb!]
  \caption{Approach overview}
  \centering
  \label{fig:approach}
  \includegraphics[width=1\textwidth]{figures/approach}
\end{figure*}

The goal of our project is to distinguish Defect Debt from Bugs. In order to do that we conducted a study using one large open source project, Google Chrome. We present our approach overview in \ref{fig:approach}. First, we extract data from the source code and from the issuer tracker repositories. Second,  we process the data to find relevant attributes and link them together. Third, we define our set of metrics and models. Then we analyze our results and answer the research questions. In the following sections we explain in details each one of the steps in our approach.   

\subsection{Data Extraction}

We use Google Chrome to conduct our study. Chrome is a large, mature open source software mostly written in C/C++. The criteria to select this project are its size, the easy access to its source code repository and to the its issuer tracker. Other than that, we had at our disposal some of the desired data that was used in a previous study. This data contained 47.938 html files and 5 tables. Each one of the files represent a bug extracted from the issue tracker. The tables contains the necessary information to link issues files, bugs and commits together. More details about the project can be found in table \ref{tab:project_details}.

To complement the data necessary to our study, we extract from the Chrome Releases website \ref{chrome_releases} the release date of each stable version release, the tag of the release and the release number. Then we store this data in our database.
 
In order to have all the source code available we clone Chromium Git repository. 

\subsection{Process data and find attributes}

After the extraction we process the data and search for attributes that will help us in our analysis. 

First, we create the attributes to store the date that the bug was reported, the release which the bug was reported, the number of comments found in the bug and the release that the bug was commited. Then we process the issues files to get the reported date and the number of comments of each bug. The release and commit dates we get from the release table. 

Second, we create the attributes to store information about the commit classification and the files included in the commit. We use Commit Guru\ref{commit_guru} to gather this information. Commit Guru is an on-line tool, which run a series of source code analysis in a given git repository url. The commit classification is based on the commit message and is used to predict the intend of a change. The process uses key words and expressions that are strong indicators of intend, like `add' or `bug fix'. The possibles classifications are `Feature Addition', `Corrective', `Preventative', `Merge', `Non Functional', `Perfective' and without classification.

Third, we create a table to store metrics values. We collect our metrics using Understand a code analysis tool. This tool can analyze different releases of a project and create a separated database containing several metrics \ref{understand_metrics} for each one of the processed releases. It is possible to access these databases through an API and collect the processed information. We checkout 18 stable releases from our Chrome repository based on the collected tag information from the selected releases. Due to the size of the project and time constraints we use only the `chrome' directory without the third party libraries in our analysis. The whole process took approximately 16 hours and generated more than 2GB of data. 

Different scripts written in Python and SQL were developed during this study in order to make this process automatic and easily reproducible. We provide the link\footnotetext{https://github.com/maldonado/soen6611} to these files just to evaluation purposes.

\subsection{Link data}

The most important data to link for our study are bug reports and the commits that are addressing them. As mentioned before, we took advantage of data used in a different study that had done this work already. Other than that, we link all of our entities in the database with releases as we are investigating the behavior of Defect Debt and Bugs per release. 

\subsection{Define metrics}




\section{Case Study Results}
\label{sec:results}
We conduct our study in a large scale open source software system to distinguish Defect Debt and Bugs. First, we analyze the history of issues in the system, and then we suggest a definition to Defect Debt and to Bugs. Based on this definition we quantify  both on them per release. Second, we analyze the evolution of Defect Debt to understand its impact on future feature addition. Third, based on our metrics we propose a model to Defect Debt.

\vspace{3mm}
\noindent\rqi
\vspace{3mm}

\noindent\textbf{Motivation:} Intuitively we know that different defects has different impacts in software quality. We hypothesize that defects in a project can be categorized in two different classifications. First, bugs which are high priority defects usually fixed near to its reported date. Second, Defect Debt which are defects that lingers in the system until an opportunity to fix it appears. Answering this research question, will provide us a way to identify and quantify Defect Debt .

\vspace{1mm}
\noindent\textbf{Approach:} To identify defect debt in a software project we first mine its issuer tracker to extract all bug reports available. Then, we link the bug report to the software releases using its reported date. Then we link the commits with bugs trough the commit message.  We analyze then the data for three different patterns, Bugs that were opened and closed in the release. Bugs that were opened in one release and fixed in a future release. Finally, bugs that were opened and was closed with a WONT\_FIX decision or that were opened and after some releases they were not addressed. (The number of releases should vary accordingly with the number of releases of the project analyzed). With the analysis done, we can quantify the results obtained for each category described.

\vspace{1mm}
\noindent\textbf{Result:} Most defects are fixed during the release that they were reported or in the next one. A number of defects reported in the current release will remain in the system for a long time, but eventually get fixed.


\begin{figure}[thb!]
    \caption{Release density evolution}
    \label{fig:release_density}
    \includegraphics[width=90mm,scale=0.5]{figures/r1}
\end{figure}

\begin{figure}[thb!]
    \caption{Release density evolution}
    \label{fig:release_density}
    \includegraphics[width=90mm,scale=0.5]{figures/r2}
\end{figure}

\vspace{3mm}
\noindent\rqii
\vspace{3mm}

\noindent\textbf{Motivation:} Technical debt can bring you short term advantage in trade off maintenance costs in the  long run. which means that when well managed technical debt can be used as a tool to achieve projects goals. That been said, we want to know how much is too much when dealing with technical debt. We argue that the amount of technical debt starts to slow down your development is the point to pay off. 

\vspace{1mm}
\noindent\textbf{Approach:} Now that we have an approach to identify defect debt we want to see how it relates with the amount of changes in the project. First we quantify the number of defect debt in the software, then we mine the source code repository to examine the changes. We will find all the changes(commits) related to a specific release using the commit date. Analyzing the commit messages we will categorize the changes of the release to see with the change is due to a bug fix or adding new features. After this analyze we will have the number of commits that were bug fix and the number of commits that were new features. To measure if  a high number of defect debt is preventing new features to be implemented in the project, we analyze if the number of new features are reducing at the same time that the number bug fixes are increasing. We also should expect that a higher number of defects is reported when the number of technical debt is higher in the project.

\vspace{1mm}
\noindent\textbf{Result:} The number of open defects increases during the project life-cycle. Defect Debt has growth peaks, but eventually get removed in future releases. Analyzing the feature addition and the defect debt graph we found that the introduction of defect debt does not impact the number of  feature addition per release. 

\begin{figure}[thb!]
    \caption{Release density evolution}
    \label{fig:release_density}
    \includegraphics[width=90mm,scale=0.5]{figures/number_of_bugs_releases}
\end{figure}

\begin{figure}[thb!]
    \caption{Release density evolution}
    \label{fig:release_density}
    \includegraphics[width=90mm,scale=0.5]{figures/technical_debt}
\end{figure}

\begin{figure}[thb!]
    \caption{Release density evolution}
    \label{fig:release_density}
    \includegraphics[width=90mm,scale=0.5]{figures/feature_addition_releases}
\end{figure}

\vspace{3mm}
\noindent\rqiii
\vspace{3mm}

\noindent\textbf{Motivation:} Learning from mistakes of the past can be useful in software engineering as well. We want to know if it is possible to determine if a defect are likely to become defect technical debt by analyzing the cases that were identified in RQ1. Answering this question will provide us with more means to effectively manage defect debt by knowing before hand that the specific defect will have to be handled in the future to avoid quality impacts. 

\vspace{1mm}
\noindent\textbf{Approach:} Based on the defect debt identified in the project we will analyze the bug reports looking for patterns in the attributes to build a prediction model. This information will serve as our training set. Then we will run the analyze again using the data from other project to measure the precision of our model. 

\vspace{1mm}
\noindent\textbf{Result:}

\section{Threats to validity}
\label{sec:threats_to_validity}
\noindent \textbf{External validity} consider the generalization of our findings. All of our findings were derived from one open source project. To minimize external validity, we chose a large open source project. That said, our results may not generalize to other open source or commercial projects.

\section{Conclusion and Future work}
\label{sec:conclusion}
In our study we analyzed Google Chrome a large open source project to understand the different characteristics between Bugs and Defect Debt. 

We show in our approach how to extract, process and link data from this project. We also explain the different metrics that we selected and how we group the metrics into releases. 

Making use of these case study setup we find that it is possible to characterize Defect Debt from bugs. Bugs are more impacting defects, and get fixed in the reported or in the next release. Whereas Defect Debt may linger in the system for a long time, but eventually it get fixed. We draw this conclusion after analyzing the behavior of the fixes throughout 25 different releases of the project. Based on the results, the average percentage of Bugs is of 69.55\%, and Defect Debt is of 30.44\%.

We found that the defect debt does not impact negatively the addition of new features. Is important to notice that we are not considering other aspects that may impact quality.  

Although the correlations between the measures are not high their directions gives us insights to add more meaningful metrics.

One possible way to improve the Defect Debt model is to calculate the effective size for the regression. Other possibility is to reformulate the problem. Changing the granularity of the analysis to the level file instead of the release level. 

In a future work we plan to include in our analyze the severity of the issues. In our current analysis we do not take this factor in consideration. This will lead us further in the understanding of Defect Debt characterization, and prevent us of relying just on the assumption that only what is import get fixed. As a matter of fact, more research in this subject can provide empirical evidence for proving or rejecting this assumption.

\bibliographystyle{IEEEtran}
\bibliography{bib}

\appendix{}
\label{sec:appendix}

\begin{table*}[!hbt]
      \begin{center}
            \caption{Metric details Part I}
            \label{tab:metrics_details_1}
            \begin{tabular}{l| c c c c c c c c}
            \toprule
            \textbf{Release}  & \textbf{corrective} & \textbf{feature\_add} & \textbf{merge} & \textbf{non\_functional} & \textbf{perfective} & \textbf{preventative}  & \textbf{not\_classified}  & \textbf{t\_churn} \\ \midrule  
                  10          &                       2878 &                         253 &                    28 &                              5 &                         54 &                            84 &                     1283 &   793600 \\
                  11          &                       2337 &                         201 &                    25 &                              4 &                         30 &                            63 &                     1121 &   477210 \\
                  12          &                       3229 &                         313 &                    39 &                             15 &                         39 &                           108 &                     1443 &   840984 \\
                  13          &                       2829 &                         255 &                    25 &                              8 &                         37 &                            86 &                     1301 &   505683 \\
                  14          &                       2493 &                         230 &                    14 &                              1 &                         32 &                            59 &                     1248 &   443070 \\
                  15          &                       3280 &                         315 &                    31 &                              8 &                         67 &                           120 &                     1684 &   775464 \\
                  16          &                       3117 &                         274 &                    41 &                              2 &                         60 &                           129 &                     1413 &   601545 \\
                  17          &                       4073 &                         367 &                    25 &                             13 &                        105 &                            86 &                     1955 &   821351 \\
                  18          &                       3631 &                         354 &                    45 &                             12 &                         43 &                           123 &                     1590 &   682760 \\
                  19          &                       3400 &                         287 &                    30 &                             16 &                         56 &                           100 &                     1506 &   629396 \\
                  20          &                       2319 &                         228 &                    12 &                             65 &                         34 &                           104 &                     1103 &   417159 \\
                  21          &                       4370 &                         398 &                    53 &                             99 &                         68 &                           133 &                     2066 &   893477 \\
                  22          &                       3133 &                         288 &                    40 &                             29 &                         51 &                           142 &                     1564 &   525595 \\
                  23          &                       4462 &                         475 &                    70 &                             28 &                         97 &                           193 &                     2642 &   879093 \\
                  24          &                       3260 &                         394 &                    46 &                             15 &                         90 &                           126 &                     1776 &   658434 \\
                  25          &                       2784 &                         278 &                    32 &                             12 &                         42 &                           103 &                     1758 &   578304 \\ \bottomrule
            \end{tabular}
      \end{center}
\end{table*}

\begin{table*}[!hbt]
      \begin{center}
            \caption{Metrics details Part II}
            \label{tab:metrics_details_2}
            \begin{tabular}{l| c c c c c c c c c c c }
            \toprule
            \textbf{Release}  & \textbf{t\_dev} & \textbf{dev\_join} & \textbf{dev\_left} & \textbf{complexity} & \textbf{loc} & \textbf{replies}  & \textbf{t\_bugfix}  & \textbf{pre\_release} & \textbf{file\_changes} & \textbf{number\_bugs} &  \textbf{defect\_debt} \\ \midrule  
                  10 &   334 &       79 &        0 &     103354 & 1054944 &            40250 &     1833 &                                631 &        28387 &        1034 &         606   \\                                                                     
                  11 &   335 &        1 &        0 &      99518 & 1032098 &            29965 &     1587 &                                552 &        20867 &         838 &         427   \\                                                                     
                  12 &   389 &       54 &        0 &      96312 &  992345 &            59356 &     2263 &                                933 &        28655 &        1257 &         763   \\                                                                     
                  13 &   396 &        7 &        0 &     100889 & 1035636 &            49727 &     2082 &                                614 &        25809 &        1097 &         561   \\                                                                     
                  14 &   381 &        0 &       15 &     105046 & 1088876 &           235710 &     1840 &                                747 &        23573 &         982 &         561   \\                                                                     
                  15 &   391 &       10 &        0 &     108763 & 1126395 &            90957 &     2814 &                                944 &        34584 &        1593 &         849   \\                                                                     
                  16 &   409 &       18 &        0 &     113591 & 1173469 &           151303 &     2670 &                               1150 &        31070 &        1423 &         789   \\                                                                     
                  17 &   432 &       23 &        0 &     116678 & 1207336 &            74159 &     3775 &                               1072 &        38762 &        2025 &         876   \\                                                                     
                  18 &   426 &        0 &        6 &     122772 & 1263954 &            97929 &     3098 &                               1015 &        32331 &        1755 &         687   \\                                                                     
                  19 &   449 &       23 &        0 &     123926 & 1287722 &            70860 &     3141 &                               1029 &        31225 &        1556 &         946   \\                                                                     
                  20 &   446 &        0 &        3 &     125966 & 1306306 &            46031 &     2174 &                                949 &        22678 &         924 &         738   \\                                                                     
                  21 &   520 &       74 &        0 &     131339 & 1374337 &            84831 &     4337 &                               1050 &        41583 &        2377 &         498   \\                                                                     
                  22 &   487 &        0 &       33 &     136731 & 1432004 &            71139 &     2927 &                                998 &        31082 &        1501 &         882   \\                                                                     
                  23 &   537 &       50 &        0 &     141480 & 1484199 &            91687 &     4614 &                                568 &        46917 &        2421 &         438   \\                                                                     
                  24 &   517 &        0 &       20 &     145514 & 1526971 &            43550 &     3541 &                                829 &        34195 &        1697 &         642   \\                                                                     
                  25 &   510 &        0 &        7 &     149835 & 1583849 &            33591 &     3019 &                                810 &        29114 &        1395 &         390   \\ \bottomrule                                                                    
            \end{tabular}
      \end{center}
\end{table*}

% \appendix{}
\label{sec:appendix_figure}

\begin{figure}[thb!]
      \caption{Fix for defects reported in release 1.}
      \label{fig:app_defect_release_1}
      \includegraphics[width=0.49\textwidth]{figures/r1}
\end{figure}

\begin{figure}[thb!]
      \caption{Fix for defects reported in release 2.}
      \label{fig:defect_release_2}
      \includegraphics[width=0.49\textwidth]{figures/r2}
\end{figure}

\begin{figure}[thb!]
      \caption{Fix for defects reported in release 3.}
      \label{fig:defect_release_3}
      \includegraphics[width=0.49\textwidth]{figures/r3}
\end{figure}

\begin{figure}[thb!]
      \caption{Fix for defects reported in release 4.}
      \label{fig:defect_release_4}
      \includegraphics[width=0.49\textwidth]{figures/r4}
\end{figure}

\begin{figure}[thb!]
      \caption{Fix for defects reported in release 5.}
      \label{fig:defect_release_5}
      \includegraphics[width=0.49\textwidth]{figures/r5}
\end{figure}

\begin{figure}[thb!]
      \caption{Fix for defects reported in release 6.}
      \label{fig:app_defect_release_6}
      \includegraphics[width=0.49\textwidth]{figures/r6}
\end{figure}

\begin{figure}[thb!]
      \caption{Fix for defects reported in release 7.}
      \label{fig:defect_release_7}
      \includegraphics[width=0.49\textwidth]{figures/r7}
\end{figure}

\begin{figure}[thb!]
      \caption{Fix for defects reported in release 8.}
      \label{fig:defect_release_8}
      \includegraphics[width=0.49\textwidth]{figures/r8}
\end{figure}

\begin{figure}[thb!]
      \caption{Fix for defects reported in release 9.}
      \label{fig:defect_release_9}
      \includegraphics[width=0.49\textwidth]{figures/r9}
\end{figure}

\begin{figure}[thb!]
      \caption{Fix for defects reported in release 10.}
      \label{fig:defect_release_10}
      \includegraphics[width=0.49\textwidth]{figures/r10}
\end{figure}

\begin{figure}[thb!]
      \caption{Fix for defects reported in release 10.}
      \label{fig:defect_release_10}
      \includegraphics[width=0.49\textwidth]{figures/r10}
\end{figure}

\begin{figure}[thb!]
      \caption{Fix for defects reported in release 10.}
      \label{fig:defect_release_10}
      \includegraphics[width=0.49\textwidth]{figures/r10}
\end{figure}

\begin{figure}[thb!]
      \caption{Fix for defects reported in release 10.}
      \label{fig:defect_release_10}
      \includegraphics[width=0.49\textwidth]{figures/r10}
\end{figure}

\begin{figure}[thb!]
      \caption{Fix for defects reported in release 10.}
      \label{fig:defect_release_10}
      \includegraphics[width=0.49\textwidth]{figures/r10}
\end{figure}

\begin{figure}[thb!]
      \caption{Fix for defects reported in release 10.}
      \label{fig:defect_release_10}
      \includegraphics[width=0.49\textwidth]{figures/r10}
\end{figure}

\begin{figure}[thb!]
      \caption{Fix for defects reported in release 10.}
      \label{fig:defect_release_10}
      \includegraphics[width=0.49\textwidth]{figures/r10}
\end{figure}

\begin{figure}[thb!]
      \caption{Fix for defects reported in release 10.}
      \label{fig:defect_release_10}
      \includegraphics[width=0.49\textwidth]{figures/r10}
\end{figure}

\begin{figure}[thb!]
      \caption{Fix for defects reported in release 10.}
      \label{fig:defect_release_10}
      \includegraphics[width=0.49\textwidth]{figures/r10}
\end{figure}


\begin{figure}[thb!]
      \caption{Fix for defects reported in release 10.}
      \label{fig:defect_release_10}
      \includegraphics[width=0.49\textwidth]{figures/r10}
\end{figure}

\begin{figure}[thb!]
      \caption{Fix for defects reported in release 10.}
      \label{fig:defect_release_10}
      \includegraphics[width=0.49\textwidth]{figures/r10}
\end{figure}

\begin{figure}[thb!]
      \caption{Fix for defects reported in release 10.}
      \label{fig:defect_release_10}
      \includegraphics[width=0.49\textwidth]{figures/r10}
\end{figure}

\begin{figure}[thb!]
      \caption{Fix for defects reported in release 10.}
      \label{fig:defect_release_10}
      \includegraphics[width=0.49\textwidth]{figures/r10}
\end{figure}

\begin{figure}[thb!]
      \caption{Fix for defects reported in release 10.}
      \label{fig:defect_release_10}
      \includegraphics[width=0.49\textwidth]{figures/r10}
\end{figure}

\begin{figure}[thb!]
      \caption{Fix for defects reported in release 10.}
      \label{fig:defect_release_10}
      \includegraphics[width=0.49\textwidth]{figures/r10}
\end{figure}

\begin{figure}[thb!]
      \caption{Fix for defects reported in release 10.}
      \label{fig:defect_release_10}
      \includegraphics[width=0.49\textwidth]{figures/r10}
\end{figure}


\end{document}
